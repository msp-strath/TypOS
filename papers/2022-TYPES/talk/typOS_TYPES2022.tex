\documentclass[xcolor=dvipsnames]{beamer}

%%%%% TODO %%%%%%%%%%%%%%%%%%%%%%%%%%%%%%%%%%%%%%%%%
%%%%%%%%%%%%%%%%%%%%%%%%%%%%%%%%%%%%%%%%%%%%%%%%%%%%

%%%%% Title etc %%%%%%%%%%%%%%%%%%%%%%%%%%%%%%%%%%%%
\title{\texorpdfstring{\centerline{\mbox{TypOS: An ``Operating System'' for Typechecking Actors}}}{TypOS: An ``Operating System'' for Typechecking Actors}}
\author{
  Guillaume Allais
  \and
  Malin Altenm\"uller
  \and
  Conor McBride
  \and
  Georgi Nakov
  \and
  \textbf{Fredrik Nordvall Forsberg}
  \and
  Craig Roy
}
\institute{
University of St Andrews,
University of Strathclyde,
and Quantinuum
}
\date[22 June 2022]{22 June 2022, TYPES, Nantes}
%%%%%%%%%%%%%%%%%%%%%%%%%%%%%%%%%%%%%%%%%%%%%%%%%%%%

%%%%% Packages %%%%%%%%%%%%%%%%%%%%%%%%%%%%%%%%%%%%%
\usepackage[skins]{tcolorbox}
\usepackage{alltt}
\usepackage[overlay, absolute]{textpos}
\usepackage{mathtools}
%%%%%%%%%%%%%%%%%%%%%%%%%%%%%%%%%%%%%%%%%%%%%%%%%%%%

%%%%%% Macros %%%%%%%%%%%%%%%%%%%%%%%%%%%%%%%%%%%%%%
\newenvironment{transbox}[1][\textwidth]{%
 \begin{tikzpicture}
 \node[text width=#1, rounded corners,
 fill=brown!20, fill opacity=0.8,text opacity=1] \bgroup%
}{\egroup;\end{tikzpicture}}
\newcommand{\titleslide}[3][\textwidth]{
\addtocounter{framenumber}{-1}%
{
  \usebackgroundtemplate{\includegraphics[height=\paperheight]{#2}}
\begin{frame}[plain]
\begin{center}
  \begin{transbox}[#1]
\begin{center}
{\Huge{{\color{black}{#3}}}}
\end{center}
  \end{transbox}
\end{center}
\end{frame}
}
}
\newcommand{\heading}[1]{{\usebeamercolor[fg]{frametitle}\textbf{#1}}}
\newcommand{\citing}[1]{{\usebeamercolor[fg]{hlightcitingcolor}#1}}
\newcommand{\hlight}[1]{{\usebeamercolor[fg]{hlighthlightcolor}#1}}

\newcommand{\circlepic}[2]{\tikz\node[circle,draw, inner sep=#1, fill overzoom image=#2] {};}

% TypOS helpers
\newcommand{\keyword}[1]{{\usebeamercolor[fg]{kwordcolor}{#1}}}
\newcommand{\syntax}{\keyword{\texttt{syntax}\;}}
\newcommand{\exec}{\texttt{exec}}
\newcommand{\InMode}[1]{\keyword{?}#1\keyword{.}}
\newcommand{\OutMode}[1]{\keyword{!}#1\keyword{.}}
\newcommand{\lsq}{\texttt{[}}
\newcommand{\rsq}{\texttt{]}}
\newcommand{\atquote}{\symbol{39}}
\newcommand{\atom}[1]{\texttt{\atquote}#1}
\newcommand{\bsl}{\texttt{\symbol{92}}}
\newcommand{\bind}[1]{\bsl #1\!\texttt{.}}

% STLC Syntax
\renewcommand{\Check}{\syntaxcat{Check}}
\newcommand{\Synth}{\syntaxcat{Synth}}
\newcommand{\Type}{\syntaxcat{Type}}

% Syntax description syntax
\newcommand{\Syntax}{\texttt{\atom{Syntax}}}
\newcommand{\Tag}{\texttt{\atom{Tag}}}
\newcommand{\Rec}{\texttt{\atom{Rec}}}
\newcommand{\Atom}{\texttt{\atom{Atom}}}
\newcommand{\Nil}{\texttt{\atom{Nil}}}
\newcommand{\Cons}{\texttt{\atom{Cons}}}
\newcommand{\NilOrCons}{\texttt{\atom{NilOrCons}}}
\newcommand{\Bind}{\keyword{\texttt{\atom{Bind}}}}
\newcommand{\Fix}{\texttt{\atom{Fix}}}
\newcommand{\EnumOrTag}{\texttt{\keyword{\atom{EnumOrTag}}}}
\newcommand{\Enum}{\texttt{\keyword{\atom{Enum}}}}
\newcommand{\Wildcard}{\texttt{\keyword{\atom{Wildcard}}}}

% STLC commands
\renewcommand{\check}{\judgement{check}}
\newcommand{\synth}{\judgement{synth}}
\newcommand{\type}{\judgement{type}}
%%%%%%%%%%%%%%%%%%%%%%%%%%%%%%%%%%%%%%%%%%%%%%%%%%%%

%%%%%% Notations %%%%%%%%%%%%%%%%%%%%%%%%%%%%%%%%%%%
\newcommand{\typosBinding}[1]{#1 \vdash}
\newcommand{\typosPushing}[3]{\textsc{#1} \vdash #2 \to #3.}
\newcommand{\typosStuckUnifying}[2]{#1 \not\sim #2}
\newcommand{\typosFailed}[1]{\text{#1}}
\newcommand{\typosAtom}[1]{\tt`#1}
\newcommand{\typosNil}{[]}
\newcommand{\typosListStart}[2]{[#1#2}
\newcommand{\typosListEnd}{]}
\newcommand{\typosListTail}[1]{\textbar #1]}
\newcommand{\typosScope}[2]{\lambda #1. #2}
\newcommand{\typosAxiom}[1]{#1}
\newcommand{\typosError}[1]{\colorbox{red}{\ensuremath{#1}}}
\newcommand{\typosDerivation}[2]{#1 \\ #2}
\newcommand{\typosBeginPrems}{\begin{array}{|@{\ }l@{}}}
\newcommand{\typosBetweenPrems}{\\}
\newcommand{\typosEndPrems}{\end{array}}
\newcommand{\typosInput}[1]{\textcolor{blue}{#1}}
\newcommand{\typosOutput}[1]{\textcolor{red}{#1}}
\newcommand{\typosCheckmark}{^\checkmark}
%%%%%%%%%%%%%%%%%%%%%%%%%%%%%%%%%%%%%%%%
\newcommand{\enumNat}[0]{\mathsf{Nat}}
\newcommand{\enumZero}[0]{\mathsf{Zero}}
\newcommand{\tagAnnForTwo}[2]{[ \mathsf{Ann} \ #1 \ #2 ]}
\newcommand{\tagAppForTwo}[2]{[ \mathsf{App} \ #1 \ #2 ]}
\newcommand{\tagArrForTwo}[2]{[ \mathsf{Arr} \ #1 \ #2 ]}
\newcommand{\tagEmbForOne}[1]{[ \mathsf{Emb} \ #1 ]}
\newcommand{\tagLamForOne}[1]{[ \mathsf{Lam} \ #1 ]}
\newcommand{\tagSuccForOne}[1]{[ \mathsf{Succ} \ #1 ]}
\newcommand{\callingtype}[1]{\textsc{type}\ #1}
\newcommand{\callingcheck}[2]{\textsc{check}\ #1 \ #2}
\newcommand{\callingsynth}[2]{\textsc{synth}\ #1 \ #2}
\usepackage{amssymb}
\usepackage{proof}

\renewcommand{\typosBinding}[1]{}
\renewcommand{\typosPushing}[3]{#2 : #3 \vdash}

\renewcommand{\typosAxiom}[1]{\infer{#1}{}}
\renewcommand{\typosDerivation}[2]{\infer{#1}{#2}}
\renewcommand{\typosBeginPrems}{}
\renewcommand{\typosBetweenPrems}{&}
\renewcommand{\typosEndPrems}{}

\renewcommand{\typosScope}[2]{\lambda #1. #2}

\renewcommand{\tagLam}[1]{#1}
\renewcommand{\tagEmb}[1]{\underline{#1}}
\renewcommand{\tagRad}[2]{(#1 : #2)}
\renewcommand{\tagApp}[2]{#1 #2}
\renewcommand{\enumNat}[0]{\mathbb{N}}
\renewcommand{\tagArr}[2]{#1 \to #2}
\renewcommand{\callingtype}[1]{\textsc{type}~ #1}
\renewcommand{\callingcheck}[2]{#1 \ni #2}
\renewcommand{\callingsynth}[2]{#1 \in #2}

%%%%%%%%%%%%%%%%%%%%%%%%%%%%%%%%%%%%%%%%%%%%%%%%%%%%


%%%%% Tweaking %%%%%%%%%%%%%%%%%%%%%%%%%%%%%%%%%%%%%
\usecolortheme[named=RawSienna]{structure}
\defbeamertemplate*{footline}{my theme}
{
  \leavevmode%
  \hbox{%
    \usebeamerfont{date in head/foot}
    \hspace*{0.97\paperwidth}\raisebox{2pt}{\insertframenumber{}}}
  \vskip0pt%
}
\beamertemplatenavigationsymbolsempty

%%% Colors %%%%%%%%%%%%%%%%%%%%%%%%%%%%%%%%%%%%%%%%%
\setbeamercolor{hlightcitingcolor}{fg=red!70!yellow}
\setbeamercolor{hlighthlightcolor}{fg=green!70!red}
\setbeamercolor{kwordcolor}{fg=red!50!blue}
%%%%%%%%%%%%%%%%%%%%%%%%%%%%%%%%%%%%%%%%%%%%%%%%%%%%

\begin{document}

\begin{frame}[noframenumbering,plain]
\titlepage
\centerline{
  \circlepic{0.8cm}{pictures/actors/gallais}
  \circlepic{0.8cm}{pictures/actors/malin}
  \circlepic{0.8cm}{pictures/actors/conor}
  \circlepic{0.8cm}{pictures/actors/georgi}
  \circlepic{0.8cm}{pictures/actors/craig-cropped}
}
\end{frame}

\section{Motivation}

\begin{frame}{A domain-specific language for typecheckers}

  An experiment in how to write typecheckers that make (more) sense.

  \bigskip

  \uncover<2->{
    Similar endeavours: Andromeda \citing{[Bauer, Haselwarter and Petkovic 2020]}, Redex \citing{[Felleisen, Findler and Flatt 2009]},  Turnstyle+ \citing{[Chang, Ballantyne, Turner and Bowman]}, \ldots

    \bigskip

    However we try to minimise demands on the order in which subproblems are solved.
  }

  \bigskip

  \uncover<3->{
  Conor McBride, 20 years ago implementing Epigram 1:
  \begin{quote}
    ``[redacted] me, I'm implementing an operating system!''
  \end{quote}
  }

  \bigskip

  \uncover<5->{
    \heading{Concrete motivation:} implementing a type theory with
    rich equational theory for free monoids and free Abelian groups.
  }

\end{frame}

\begin{frame}{Why not just a shallow embedding?}

  \uncover<2->{
    \heading{Logical Framework aspects:} we implement syntax with binding once, and then it Just Works.
  }

  \bigskip

  \uncover<3->{
    \heading{Resumptions should be updatable:} progress might have happened while a process was asleep.
  }

  \bigskip

  \uncover<4->{
    \heading{Ruling out design errors by construction:} a first-order representation means we can do static analysis on the typecheckers themselves.
  }

\end{frame}

\titleslide[15em]{pictures/globe}{A Tour of TypOS}

\section{A Tour of TypOS}

\subsection{Syntax descriptions}

\begin{frame}{Syntax descriptions}

  \uncover<2->{
    We support a Lisp-style generic syntax for terms:
    \begin{itemize}
    \item atoms \atquote{}$a$
    \item cons lists \lsq $t_0$ $t_1$ \ldots\ $t_n$\rsq
    \item variables $x$ and bindings $\bind x t$
    \end{itemize}
  }

  \bigskip

  \uncover<3->{
  Simple and uniform to write and parse.
  }

  \bigskip

  \uncover<4->{ Users can restrict the shape of terms using
    context-free \hlight{syntax descriptions}.
    \uncover<5->{We always offer a \texttt{Wildcard} description allowing anything.}
  }

  \bigskip

  \uncover<6->{There is a syntax description of syntax descriptions, which we use to check syntax descriptions.}

\end{frame}

%\begin{frame}[fragile]{A syntax description for bidirectional STLC}
%
%\begin{alltt}
%\syntax
%  \keyword{\{} 'Type  = [\EnumOrTag  ['Nat] [ ['Arr 'Type 'Type]]]
%  \keyword{;} 'Check = [\EnumOrTag [] [ ['Lam [\Bind 'Synth 'Check]]
%                              ['Emb 'Synth]]]
%  \keyword{;} 'Synth = [\EnumOrTag [] [ ['Ann 'Check 'Type]
%                              ['App 'Synth 'Check]]]
%  \keyword{\}}
%\end{alltt}
%
%\end{frame}

\subsection{Judgement forms as interaction protocols}

\begin{frame}[fragile]{Judgement forms as interaction protocols}

  We recast the notion of \fbox{judgement form} as
  \hlight{communication protocol}:

  \begin{itemize}
  \item What to communicate (of what syntax description)?
    \smallskip
  \item In which direction (input or output)?
  \end{itemize}

  \bigskip

  \uncover<2->{
    A basic form of session types \citing{[Honda 1993]}.
  }

  \bigskip

  \uncover<3->{
    For example:
    \begin{alltt}
      \ \ \ type : \InMode{'Type} \\
      \ \ \ check : \InMode{'Type} \InMode{'Check} \\
      \ \ \ synth : \InMode{'Synth} \OutMode{'Type}
    \end{alltt}
    }
    \end{frame}

% TODO: expression vs pattern distinction

\subsection{Typing rules as actors}

\begin{frame}[fragile]{Typing rules as actors}

``A rule is a server for its conclusion, and a client for its premises.''

\bigskip

\uncover<2->{

  That is: typing rules are implemented by actors, which

  \begin{itemize}
  \item must fulfill their protocol with respect to their parent;
    \smallskip
  \item typically spawns children processes for all its premises.
  \end{itemize}
}

\bigskip

\uncover<3->{
  Inspired by the actor model \citing{[Hewitt, Bishop and Steiger 1973]} of concurrent programming.
}

\bigskip

\uncover<4->{
  Typechecking process $actor$ with parent channel $p$ is defined by
\begin{alltt}
\ \ \ actor@p = ...
\end{alltt}
}

\end{frame}

\begin{frame}{Actor constructs: winning}

  \begin{center}
    \begin{tabular}{c}
      \scalebox{4}{\textvisiblespace}
      \\[1em]
      a successful, finished actor
    \end{tabular}
  \end{center}

  \bigskip

  (Victory is silent.)

\end{frame}

\begin{frame}{Actor constructs: failing}

  \begin{center}
    \begin{tabular}{c}
      \scalebox{2}{\texttt{\# "error message"}}
      \\[1em]
      an unsuccessful, finished actor
    \end{tabular}
  \end{center}

\end{frame}

\begin{frame}{Actor constructs: printing}

  \begin{center}
    \begin{tabular}{c}
      \scalebox{2}{\texttt{PRINTF "message text".}}
      \\[1em]
      printing a message before continuing
    \end{tabular}
  \end{center}

\end{frame}

\begin{frame}{Actor constructs: generating fresh meta variables}

  \begin{center}
    \begin{tabular}{c}
      \scalebox{2}{\texttt{$sd?X$.}}
      \\[1em]
      generate a fresh meta $X$ of syntax description $sd$
    \end{tabular}
  \end{center}

  \bigskip

  \uncover<2->{
    Meta variables stand for \emph{unknown} terms.
  }

\end{frame}

\begin{frame}{Actor constructs: matching on terms}

  \begin{center}
    \begin{tabular}{c}
      \scalebox{2}{\texttt{case} t \texttt{\{ $p_1$ \texttt{->} $a_1$ \texttt{;} \ldots \texttt{\}}}}
      \\[1em]
      match term $t$ against patterns $p_i$; continue as actor $a_i$ when matching
    \end{tabular}
  \end{center}

  \bigskip

  \uncover<2->{
    Blocks if $t$ is a metavariable.
  }

\end{frame}

\begin{frame}{Actor constructs: forking}

  \begin{center}
    \begin{tabular}{c}
      \scalebox{2}{\texttt{a | b}}
      \\[1em]
      continue as \texttt{a} and \texttt{b} in parallel
    \end{tabular}
  \end{center}

  \bigskip

  \uncover<2->{
    Progress in \texttt{b} might enable further progress in \texttt{a} and vice versa.
  }

\end{frame}

\begin{frame}{Actor constructs: declaring constraints}

  \begin{center}
    \begin{tabular}{c}
      \scalebox{2}{$t_1 \sim t_2$}
      \\[1em]
      make $t_1$ unify with $t_2$
    \end{tabular}
  \end{center}

\end{frame}

\begin{frame}{Actor constructs: spawning children}

  \begin{center}
    \begin{tabular}{c}
      \scalebox{2}{$actor$\texttt{@}$p$.}
      \\[1em]
      spawn a new child $actor$ on channel $p$
    \end{tabular}
  \end{center}

\end{frame}

\begin{frame}{Actor constructs: sending and receiving messages}

  \begin{center}
    \begin{tabular}{c}
      \scalebox{2}{$p$\texttt{!}$t$.}
      \\[1em]
      send term $t$ on channel $p$
    \end{tabular}
  \end{center}

  \uncover<2->{
  \begin{center}
    \begin{tabular}{c}
      \scalebox{2}{$p$\texttt{?}$t$.}
      \\[1em]
      receive term $t$ on channel $p$
    \end{tabular}
  \end{center}
  }

  \bigskip

  \uncover<3->{
    Messages must conform to $p$'s protocol.
  }

\end{frame}

\begin{frame}{Actor constructs: binding local variables}

  \begin{center}
    \begin{tabular}{c}
      \scalebox{2}{\texttt{\textbackslash} $x$.}
      \\[1em]
      bring fresh object variable $x$ into scope
    \end{tabular}
  \end{center}

\end{frame}

\begin{frame}{Actor constructs: extending local contexts}

  \begin{center}
    \begin{tabular}{c}
      \scalebox{2}{$ctx$ \texttt{|-} $x$ \texttt{->} $t$}
      \\[1em]
      extend declared context $ctx$ to map object variable $x$ to term $t$
    \end{tabular}
  \end{center}

\end{frame}

\begin{frame}{Actor constructs: querying local contexts}

  \begin{center}
    \begin{tabular}{c}
      \scalebox{2}{\texttt{if} $x$ \texttt{in} $ctx$ \texttt{\{} $t$ \texttt{->} $a$ \texttt{\}} \texttt{else} $b$}
      \\[1em]
      Look up variable $x$ in declared context $ctx$; \\ if found, bind associated value as $t$ and continue as $a$, \\ otherwise continue as $b$
    \end{tabular}
  \end{center}

\end{frame}

\begin{frame}[fragile]{Actors for bidirectional type checking of STLC}

\begin{verbatim}
check@p = p?ty. p?tm. case tm
  { ['Lam \x. body] -> 'Type?S. 'Type?T.
     ( ty ~ ['Arr S T]
     | \x. ctxt |- x -> S. check@q. q!T. q!body.)
  ; ['Emb e] -> synth@q. q!e. q?S. S ~ ty }

synth@p = p?tm. if tm in ctxt
  { S -> p!S. }
  else case tm
  { ['Ann t T] -> ( type@q. q!T.
                  | check@r. r!T. r!t.
                  | p!T. )
  ; ['App f s] -> 'Type?S. 'Type?T. p!T.
     ( synth@q. q!f. q?F. F ~ ['Arr S T]
     | check@r. r!S. r!s.) }
\end{verbatim}

\end{frame}

\subsection{Executing actors}

\begin{frame}{Executing actors}

  \uncover<2->{
    We currently run actors on a stack-based virtual machine.
  }

  \bigskip

  \uncover<3->{
    We run each actor until it blocks, and then try the next one, until execution stabilises.
  }

  \bigskip

  \uncover<4->{
    Metavariables are shared, which is okay, since they are updated monotonically \citing{[Kuper 2015]}.
  }

  \bigskip

  \uncover<5->{
    We can extract a typing derivation from the final configuration of the stack.
  }

\end{frame}

\section{Examples}

\titleslide[14em]{pictures/graslin}{Some examples}

\begin{frame}[fragile]{\texttt{typos --latex=stlc.tex stlc.act}}
$\displaystyle
\begin{array}{l}
\typosDerivation{\callingcheck{\typosInput{\tagArrForTwo{\enumNat}{\enumNat}}}{\typosInput{\tagLamForOne{\typosScope{z}{\tagEmbForOne{\tagAppForTwo{\tagAnnForTwo{\tagLamForOne{\typosScope{\_}{?u}}}{\tagArrForTwo{\enumNat}{\enumNat}}}{\tagEmbForOne{\mathit{z}}}}}}}} }
 {\typosBeginPrems
  \typosDerivation{\typosBinding{z_0}\typosPushing{ctxt}{z_0}{\enumNat}}
   {\typosBeginPrems
    \typosDerivation{\callingcheck{\typosInput{\enumNat}}{\typosInput{\tagEmbForOne{\tagAppForTwo{\tagAnnForTwo{\tagLamForOne{\typosScope{\_}{?u}}}{\tagArrForTwo{\enumNat}{\enumNat}}}{\tagEmbForOne{\mathit{z_0}}}}}} }
     {\typosBeginPrems
      \typosDerivation{\callingsynth{\typosInput{\tagAppForTwo{\tagAnnForTwo{\tagLamForOne{\typosScope{\_}{?u}}}{\tagArrForTwo{\enumNat}{\enumNat}}}{\tagEmbForOne{\mathit{z_0}}}}}{\typosOutput{\enumNat}} }
       {\typosBeginPrems
        \typosDerivation{\callingsynth{\typosInput{\tagAnnForTwo{\tagLamForOne{\typosScope{\_}{?u}}}{\tagArrForTwo{\enumNat}{\enumNat}}}}{\typosOutput{\tagArrForTwo{\enumNat}{\enumNat}}} }
         {\typosBeginPrems
          \typosAxiom{\callingtype{\typosInput{\tagArrForTwo{\enumNat}{\enumNat}}} \typosCheckmark}
          \typosBetweenPrems
          \typosDerivation{\callingcheck{\typosInput{\tagArrForTwo{\enumNat}{\enumNat}}}{\typosInput{\tagLamForOne{\typosScope{\_}{?u}}}} }
           {\typosBeginPrems
            \typosDerivation{\typosBinding{w_1}\typosPushing{ctxt}{w_1}{\enumNat}}
             {\typosBeginPrems
              \typosAxiom{\callingcheck{\typosInput{\enumNat}}{\typosInput{?u}} }
              \typosEndPrems}
            \typosEndPrems}
          \typosEndPrems}
        \typosBetweenPrems
        \typosDerivation{\callingcheck{\typosInput{\enumNat}}{\typosInput{\tagEmbForOne{\mathit{z_0}}}} \typosCheckmark}
         {\typosBeginPrems
          \typosAxiom{\callingsynth{\typosInput{\mathit{z_0}}}{\typosOutput{\enumNat}} \typosCheckmark}
          \typosEndPrems}
        \typosEndPrems}
      \typosEndPrems}
    \typosEndPrems}
  \typosEndPrems}
\end{array}
$
\end{frame}

\begin{frame}[fragile]{\texttt{typos --latex=stlc.tex stlc.act} completed}
$\displaystyle
\begin{array}{l}
\typosDerivation{\callingcheck{\typosInput{\tagArrForTwo{\enumNat}{\enumNat}}}{\typosInput{\tagLamForOne{\typosScope{z}{\tagEmbForOne{\tagAppForTwo{\tagAnnForTwo{\tagLamForOne{\typosScope{\_}{\tagSuccForOne{\enumZero}}}}{\tagArrForTwo{\enumNat}{\enumNat}}}{\tagEmbForOne{\mathit{z}}}}}}}} \typosCheckmark}
 {\typosBeginPrems
  \typosDerivation{\typosBinding{z_0}\typosPushing{ctxt}{z_0}{\enumNat}}
   {\typosBeginPrems
    \typosDerivation{\callingcheck{\typosInput{\enumNat}}{\typosInput{\tagEmbForOne{\tagAppForTwo{\tagAnnForTwo{\tagLamForOne{\typosScope{\_}{\tagSuccForOne{\enumZero}}}}{\tagArrForTwo{\enumNat}{\enumNat}}}{\tagEmbForOne{\mathit{z_0}}}}}} \typosCheckmark}
     {\typosBeginPrems
      \typosDerivation{\callingsynth{\typosInput{\tagAppForTwo{\tagAnnForTwo{\tagLamForOne{\typosScope{\_}{\tagSuccForOne{\enumZero}}}}{\tagArrForTwo{\enumNat}{\enumNat}}}{\tagEmbForOne{\mathit{z_0}}}}}{\typosOutput{\enumNat}} \typosCheckmark}
       {\typosBeginPrems
        \typosDerivation{\callingsynth{\typosInput{\tagAnnForTwo{\tagLamForOne{\typosScope{\_}{\tagSuccForOne{\enumZero}}}}{\tagArrForTwo{\enumNat}{\enumNat}}}}{\typosOutput{\tagArrForTwo{\enumNat}{\enumNat}}} \typosCheckmark}
         {\typosBeginPrems
          \typosAxiom{\callingtype{\typosInput{\tagArrForTwo{\enumNat}{\enumNat}}} \typosCheckmark}
          \typosBetweenPrems
          \typosDerivation{\callingcheck{\typosInput{\tagArrForTwo{\enumNat}{\enumNat}}}{\typosInput{\tagLamForOne{\typosScope{\_}{\tagSuccForOne{\enumZero}}}}} \typosCheckmark}
           {\typosBeginPrems
            \typosDerivation{\typosBinding{w_1}\typosPushing{ctxt}{w_1}{\enumNat}}
             {\typosBeginPrems
              \typosDerivation{\callingcheck{\typosInput{\enumNat}}{\typosInput{\tagSuccForOne{\enumZero}}} \typosCheckmark}
               {\typosBeginPrems
                \typosAxiom{\callingcheck{\typosInput{\enumNat}}{\typosInput{\enumZero}} \typosCheckmark}
                \typosEndPrems}
              \typosEndPrems}
            \typosEndPrems}
          \typosEndPrems}
        \typosBetweenPrems
        \typosDerivation{\callingcheck{\typosInput{\enumNat}}{\typosInput{\tagEmbForOne{\mathit{z_0}}}} \typosCheckmark}
         {\typosBeginPrems
          \typosAxiom{\callingsynth{\typosInput{\mathit{z_0}}}{\typosOutput{\enumNat}} \typosCheckmark}
          \typosEndPrems}
        \typosEndPrems}
      \typosEndPrems}
    \typosEndPrems}
  \typosEndPrems}
\end{array}
$
\end{frame}

\begin{frame}[fragile]{\texttt{typos --latex-animated=stlc-ann.tex stlc.act}}
$\displaystyle
\begin{array}{l}
\visible<0->{\typosDerivation{\callingcheck{\typosInput{\uncover<0-23>{\tagArrForTwo{\enumNat}{\enumNat}}}}{\typosInput{\uncover<1-23>{\tagLamForOne{\typosScope{z}{\tagEmbForOne{\tagAppForTwo{\tagAnnForTwo{\tagLamForOne{\typosScope{\_}{\tagSuccForOne{\enumZero}}}}{\tagArrForTwo{\enumNat}{\enumNat}}}{\tagEmbForOne{\mathit{z}}}}}}}}} \mathrlap{}
                                                                                                                                                                                                                                                                                                                                              \mathrlap{}
                                                                                                                                                                                                                                                                                                                                              \mathrlap{}
                                                                                                                                                                                                                                                                                                                                              \mathrlap{}
                                                                                                                                                                                                                                                                                                                                              \mathrlap{}
                                                                                                                                                                                                                                                                                                                                              \mathrlap{}
                                                                                                                                                                                                                                                                                                                                              \mathrlap{}
                                                                                                                                                                                                                                                                                                                                              \mathrlap{}
                                                                                                                                                                                                                                                                                                                                              \mathrlap{}
                                                                                                                                                                                                                                                                                                                                              \mathrlap{}
                                                                                                                                                                                                                                                                                                                                              \mathrlap{}
                                                                                                                                                                                                                                                                                                                                              \mathrlap{}
                                                                                                                                                                                                                                                                                                                                              \mathrlap{}
                                                                                                                                                                                                                                                                                                                                              \mathrlap{}
                                                                                                                                                                                                                                                                                                                                              \mathrlap{}
                                                                                                                                                                                                                                                                                                                                              \mathrlap{}
                                                                                                                                                                                                                                                                                                                                              \mathrlap{}
                                                                                                                                                                                                                                                                                                                                              \mathrlap{}
                                                                                                                                                                                                                                                                                                                                              \mathrlap{}
                                                                                                                                                                                                                                                                                                                                              \mathrlap{}
                                                                                                                                                                                                                                                                                                                                              \mathrlap{}
                                                                                                                                                                                                                                                                                                                                              \mathrlap{}
                                                                                                                                                                                                                                                                                                                                              \mathrlap{}
                                                                                                                                                                                                                                                                                                                                              \uncover<23-23>{\typosCheckmark}}
              {\typosBeginPrems
               \visible<2->{\typosDerivation{\typosBinding{z_0}\typosPushing{ctxt}{z_0}{\uncover<3-23>{\enumNat}}}
                             {\typosBeginPrems
                              \visible<3->{\typosDerivation{\callingcheck{\typosInput{\uncover<3-23>{\enumNat}}}{\typosInput{\uncover<4-23>{\tagEmbForOne{\tagAppForTwo{\tagAnnForTwo{\tagLamForOne{\typosScope{\_}{\tagSuccForOne{\enumZero}}}}{\tagArrForTwo{\enumNat}{\enumNat}}}{\tagEmbForOne{\mathit{z_0}}}}}}} \mathrlap{}
                                                                                                                                                                                                                                                                                                                      \mathrlap{}
                                                                                                                                                                                                                                                                                                                      \mathrlap{}
                                                                                                                                                                                                                                                                                                                      \mathrlap{}
                                                                                                                                                                                                                                                                                                                      \mathrlap{}
                                                                                                                                                                                                                                                                                                                      \mathrlap{}
                                                                                                                                                                                                                                                                                                                      \mathrlap{}
                                                                                                                                                                                                                                                                                                                      \mathrlap{}
                                                                                                                                                                                                                                                                                                                      \mathrlap{}
                                                                                                                                                                                                                                                                                                                      \mathrlap{}
                                                                                                                                                                                                                                                                                                                      \mathrlap{}
                                                                                                                                                                                                                                                                                                                      \mathrlap{}
                                                                                                                                                                                                                                                                                                                      \mathrlap{}
                                                                                                                                                                                                                                                                                                                      \mathrlap{}
                                                                                                                                                                                                                                                                                                                      \mathrlap{}
                                                                                                                                                                                                                                                                                                                      \mathrlap{}
                                                                                                                                                                                                                                                                                                                      \mathrlap{}
                                                                                                                                                                                                                                                                                                                      \mathrlap{}
                                                                                                                                                                                                                                                                                                                      \mathrlap{}
                                                                                                                                                                                                                                                                                                                      \mathrlap{}
                                                                                                                                                                                                                                                                                                                      \uncover<23-23>{\typosCheckmark}}
                                            {\typosBeginPrems
                                             \visible<5->{\typosDerivation{\callingsynth{\typosInput{\uncover<5-23>{\tagAppForTwo{\tagAnnForTwo{\tagLamForOne{\typosScope{\_}{\tagSuccForOne{\enumZero}}}}{\tagArrForTwo{\enumNat}{\enumNat}}}{\tagEmbForOne{\mathit{z_0}}}}}}{\typosOutput{\mathrlap{\uncover<6-18>{\mathit{???}}}
                                                                                                                                                                                                                                                                                            \uncover<19-23>{\enumNat}}} \mathrlap{}
                                                                                                                                                                                                                                                                                                                        \mathrlap{}
                                                                                                                                                                                                                                                                                                                        \mathrlap{}
                                                                                                                                                                                                                                                                                                                        \mathrlap{}
                                                                                                                                                                                                                                                                                                                        \mathrlap{}
                                                                                                                                                                                                                                                                                                                        \mathrlap{}
                                                                                                                                                                                                                                                                                                                        \mathrlap{}
                                                                                                                                                                                                                                                                                                                        \mathrlap{}
                                                                                                                                                                                                                                                                                                                        \mathrlap{}
                                                                                                                                                                                                                                                                                                                        \mathrlap{}
                                                                                                                                                                                                                                                                                                                        \mathrlap{}
                                                                                                                                                                                                                                                                                                                        \mathrlap{}
                                                                                                                                                                                                                                                                                                                        \mathrlap{}
                                                                                                                                                                                                                                                                                                                        \mathrlap{}
                                                                                                                                                                                                                                                                                                                        \mathrlap{}
                                                                                                                                                                                                                                                                                                                        \mathrlap{}
                                                                                                                                                                                                                                                                                                                        \mathrlap{}
                                                                                                                                                                                                                                                                                                                        \mathrlap{}
                                                                                                                                                                                                                                                                                                                        \uncover<23-23>{\typosCheckmark}}
                                                           {\typosBeginPrems
                                                            \visible<7->{\typosDerivation{\callingsynth{\typosInput{\uncover<7-23>{\tagAnnForTwo{\tagLamForOne{\typosScope{\_}{\tagSuccForOne{\enumZero}}}}{\tagArrForTwo{\enumNat}{\enumNat}}}}}{\typosOutput{\uncover<18-23>{\tagArrForTwo{\enumNat}{\enumNat}}}} \mathrlap{}
                                                                                                                                                                                                                                                                                                                    \mathrlap{}
                                                                                                                                                                                                                                                                                                                    \mathrlap{}
                                                                                                                                                                                                                                                                                                                    \mathrlap{}
                                                                                                                                                                                                                                                                                                                    \mathrlap{}
                                                                                                                                                                                                                                                                                                                    \mathrlap{}
                                                                                                                                                                                                                                                                                                                    \mathrlap{}
                                                                                                                                                                                                                                                                                                                    \mathrlap{}
                                                                                                                                                                                                                                                                                                                    \mathrlap{}
                                                                                                                                                                                                                                                                                                                    \mathrlap{}
                                                                                                                                                                                                                                                                                                                    \mathrlap{}
                                                                                                                                                                                                                                                                                                                    \mathrlap{}
                                                                                                                                                                                                                                                                                                                    \mathrlap{\uncover<22-22>{\typosCheckmark}}
                                                                                                                                                                                                                                                                                                                    \mathrlap{\uncover<21-21>{\typosCheckmark}}
                                                                                                                                                                                                                                                                                                                    \mathrlap{\uncover<20-20>{\typosCheckmark}}
                                                                                                                                                                                                                                                                                                                    \mathrlap{\uncover<19-19>{\typosCheckmark}}
                                                                                                                                                                                                                                                                                                                    \uncover<23-23>{\typosCheckmark}}
                                                                          {\typosBeginPrems
                                                                           \visible<8->{\typosAxiom{\callingtype{\typosInput{\uncover<8-23>{\tagArrForTwo{\enumNat}{\enumNat}}}} \mathrlap{}
                                                                                                                                                                                 \mathrlap{}
                                                                                                                                                                                 \mathrlap{}
                                                                                                                                                                                 \mathrlap{\uncover<22-22>{\typosCheckmark}}
                                                                                                                                                                                 \mathrlap{\uncover<21-21>{\typosCheckmark}}
                                                                                                                                                                                 \mathrlap{\uncover<20-20>{\typosCheckmark}}
                                                                                                                                                                                 \mathrlap{\uncover<19-19>{\typosCheckmark}}
                                                                                                                                                                                 \mathrlap{\uncover<18-18>{\typosCheckmark}}
                                                                                                                                                                                 \mathrlap{\uncover<17-17>{\typosCheckmark}}
                                                                                                                                                                                 \mathrlap{\uncover<16-16>{\typosCheckmark}}
                                                                                                                                                                                 \mathrlap{\uncover<15-15>{\typosCheckmark}}
                                                                                                                                                                                 \mathrlap{\uncover<14-14>{\typosCheckmark}}
                                                                                                                                                                                 \mathrlap{\uncover<13-13>{\typosCheckmark}}
                                                                                                                                                                                 \mathrlap{\uncover<12-12>{\typosCheckmark}}
                                                                                                                                                                                 \mathrlap{\uncover<11-11>{\typosCheckmark}}
                                                                                                                                                                                 \uncover<23-23>{\typosCheckmark}}}
                                                                           \typosBetweenPrems
                                                                           \visible<11->{\typosDerivation{\callingcheck{\typosInput{\uncover<11-23>{\tagArrForTwo{\enumNat}{\enumNat}}}}{\typosInput{\uncover<12-23>{\tagLamForOne{\typosScope{\_}{\tagSuccForOne{\enumZero}}}}}} \mathrlap{}
                                                                                                                                                                                                                                                                                  \mathrlap{}
                                                                                                                                                                                                                                                                                  \mathrlap{}
                                                                                                                                                                                                                                                                                  \mathrlap{}
                                                                                                                                                                                                                                                                                  \mathrlap{}
                                                                                                                                                                                                                                                                                  \mathrlap{}
                                                                                                                                                                                                                                                                                  \mathrlap{}
                                                                                                                                                                                                                                                                                  \mathrlap{\uncover<22-22>{\typosCheckmark}}
                                                                                                                                                                                                                                                                                  \mathrlap{\uncover<21-21>{\typosCheckmark}}
                                                                                                                                                                                                                                                                                  \mathrlap{\uncover<20-20>{\typosCheckmark}}
                                                                                                                                                                                                                                                                                  \mathrlap{\uncover<19-19>{\typosCheckmark}}
                                                                                                                                                                                                                                                                                  \mathrlap{\uncover<18-18>{\typosCheckmark}}
                                                                                                                                                                                                                                                                                  \uncover<23-23>{\typosCheckmark}}
                                                                                          {\typosBeginPrems
                                                                                           \visible<13->{\typosDerivation{\typosBinding{w_1}\typosPushing{ctxt}{w_1}{\uncover<14-23>{\enumNat}}}
                                                                                                          {\typosBeginPrems
                                                                                                           \visible<14->{\typosDerivation{\callingcheck{\typosInput{\uncover<14-23>{\enumNat}}}{\typosInput{\uncover<15-23>{\tagSuccForOne{\enumZero}}}} \mathrlap{}
                                                                                                                                                                                                                                                         \mathrlap{}
                                                                                                                                                                                                                                                         \mathrlap{}
                                                                                                                                                                                                                                                         \mathrlap{}
                                                                                                                                                                                                                                                         \mathrlap{\uncover<22-22>{\typosCheckmark}}
                                                                                                                                                                                                                                                         \mathrlap{\uncover<21-21>{\typosCheckmark}}
                                                                                                                                                                                                                                                         \mathrlap{\uncover<20-20>{\typosCheckmark}}
                                                                                                                                                                                                                                                         \mathrlap{\uncover<19-19>{\typosCheckmark}}
                                                                                                                                                                                                                                                         \mathrlap{\uncover<18-18>{\typosCheckmark}}
                                                                                                                                                                                                                                                         \uncover<23-23>{\typosCheckmark}}
                                                                                                                          {\typosBeginPrems
                                                                                                                           \visible<16->{\typosAxiom{\callingcheck{\typosInput{\uncover<16-23>{\enumNat}}}{\typosInput{\uncover<17-23>{\enumZero}}} \mathrlap{}
                                                                                                                                                                                                                                                    \mathrlap{}
                                                                                                                                                                                                                                                    \mathrlap{\uncover<22-22>{\typosCheckmark}}
                                                                                                                                                                                                                                                    \mathrlap{\uncover<21-21>{\typosCheckmark}}
                                                                                                                                                                                                                                                    \mathrlap{\uncover<20-20>{\typosCheckmark}}
                                                                                                                                                                                                                                                    \mathrlap{\uncover<19-19>{\typosCheckmark}}
                                                                                                                                                                                                                                                    \mathrlap{\uncover<18-18>{\typosCheckmark}}
                                                                                                                                                                                                                                                    \uncover<23-23>{\typosCheckmark}}}
                                                                                                                           \typosEndPrems}}
                                                                                                           \typosEndPrems}}
                                                                                           \typosEndPrems}}
                                                                           \typosEndPrems}}
                                                            \typosBetweenPrems
                                                            \visible<19->{\typosDerivation{\callingcheck{\typosInput{\uncover<19-23>{\enumNat}}}{\typosInput{\uncover<20-23>{\tagEmbForOne{\mathit{z_0}}}}} \mathrlap{}
                                                                                                                                                                                                            \mathrlap{}
                                                                                                                                                                                                            \mathrlap{}
                                                                                                                                                                                                            \mathrlap{}
                                                                                                                                                                                                            \uncover<23-23>{\typosCheckmark}}
                                                                           {\typosBeginPrems
                                                                            \visible<21->{\typosAxiom{\callingsynth{\typosInput{\uncover<21-23>{\mathit{z_0}}}}{\typosOutput{\uncover<22-23>{\enumNat}}} \mathrlap{}
                                                                                                                                                                                                         \mathrlap{}
                                                                                                                                                                                                         \uncover<23-23>{\typosCheckmark}}}
                                                                            \typosEndPrems}}
                                                            \typosEndPrems}}
                                             \typosEndPrems}}
                              \typosEndPrems}}
               \typosEndPrems}}
\end{array}
$

\end{frame}

\section{Verification of actors}

\begin{frame}{Verification of actors}

  What do we get by construction?

  \bigskip

  \begin{itemize}
  \item<2-> Protocols and modes \hlight{$\implies$} rely/guarantee contracts
    \smallskip
  \item<3-> Actors only knowing about free variables they themselves create \hlight{$\implies$} stability under substitution
    \smallskip
  \item<4-> ``Schematic variables'' have one explicit binding site \hlight{$\implies$} scopes are not escaped
    \smallskip
  \item<5-> \ldots
  \end{itemize}

\end{frame}

\section{Summary and future work}

\begin{frame}{Summary and future work}

  TypOS is an domain-specific language for writing typecheckers.

  \bigskip

  Judgements have \hlight{modes} (input/output protocols), typing rules are \hlight{actors} (spawning and communicating with children).

  \bigskip

  A wide range of typechecking, evaluation and elaboration processes
  can be implemented this way.

  \bigskip

  \heading{In the future:} a truly concurrent runtime.

  \bigskip

  \begin{center}
    {\Large \url{https://github.com/msp-strath/TypOS}}
  \end{center}


  \only<2>{
    \TPshowboxestrue
    \TPMargin{6pt}
    \textblockcolor{brown!15}
    \setlength{\TPboxrulesize}{2pt}
    \begin{textblock}{6.2}(5,6)
      { \Huge
        \usebeamercolor[fg]{frametitle}
        \centerline{Thank you!}
      }
    \end{textblock}
  }

\end{frame}


\section{References}

\begin{frame}{References}
\framesubtitle{In order of appearance}

\footnotesize

\begin{itemize}
  \item  \citing{Andrej Bauer, Philipp G. Haselwarter, and Anja Petkovic}. Equality checking for general type theories in Andromeda 2. In Anna Maria Bigatti, Jacques Carette, James H. Davenport, Michael Joswig, and Timo de Wolff, editors, \textit{ICMS ’20}, pages 253--259. Springer, 2020.
  \item \citing{Matthias Felleisen, Robert Bruce Findler, and Matthew Flatt}. \textit{Semantics Engineering with PLT Redex}. MIT Press, 2009.
  \item \citing{Stephen Chang, Michael Ballantyne, Milo Turner, and William J. Bowman.} Dependent type systems as macros. \textit{Proc. ACM Program. Lang.}, 4(POPL):3:1--3:29, 2020.
  \item  \citing{Kohei Honda}. Types for dyadic interaction. In Eike Best, editor, \textit{CONCUR ’93}, pages 509--523. Springer, 1993.
  \item  \citing{Carl Hewitt, Peter Bishop, and Richard Steiger}. A universal modular actor formalism for artificial intelligence. In \textit{IJCAI ’73}, pages 235--245. Morgan Kaufmann Publishers, 1973.
  \item  \citing{Lindsey Kuper}. \emph{Lattice-Based Data Structures For Deterministic Parallel And Distributed Programming}. PhD thesis, Indiana University, 2015.
\end{itemize}


{ \tiny
\textbf{Image credits:}
\begin{itemize}
\item ``Shakespeare's Globe Theatre, London'' by Neil Howard, \url{https://flic.kr/p/LtqfmA}, CC BY-NC 2.0
\item ``L'Op\'era Graslin (Le Voyage \`a Nantes)'' by Jean-Pierre Dalbéra, \url{https://flic.kr/p/f9BB5h}, CC BY 2.0
\end{itemize}
}
\end{frame}

\end{document}
