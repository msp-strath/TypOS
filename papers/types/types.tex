\documentclass{easychair}

\usepackage{xspace}
\usepackage{textcomp}
\usepackage{hyperref}
\usepackage{microtype}
\usepackage{wrapfig}
\setlength{\intextsep}{0pt}%
% \setlength{\columnsep}{2pt}%
\setlength{\abovedisplayshortskip}{1.5pt}
\setlength{\belowdisplayshortskip}{1.5pt}

\lstset{basicstyle=\tt}
\lstset{
    literate={~}{\raisebox{0.5ex}{\tt\texttildelow}}{1}
}
% Name of the project
\newcommand{\TypOS}{TypOS\xspace}

% TypOS helpers
\newcommand{\syntax}{\texttt{syntax}\;}
\newcommand{\exec}{\texttt{exec}}
\newcommand{\nil}{\texttt{[]}}
\newcommand{\lsq}{\texttt{[}}
\newcommand{\pipe}{\texttt{|}}
\newcommand{\rsq}{\texttt{]}}
\newcommand{\atom}[1]{\texttt{\symbol{39}}#1\;}
\newcommand{\bsl}{\texttt{\symbol{92}}}
\newcommand{\bind}[1]{\bsl #1\!\texttt{.}}
\newcommand{\listof}[1]{\lsq #1 \rsq}
\newcommand{\braced}[1]{$\lbrace$ #1 $\rbrace$}
\newcommand{\hab}{\;\texttt{:}\;}
\newcommand{\bang}{\texttt{!}}
\newcommand{\query}{\texttt{?}}
\newcommand{\channel}[1]{\texttt{#1}}
\newcommand{\case}[1]{\texttt{case}}
\newcommand{\caselabel}[1]{\texttt{/#1}}
\newcommand{\lookupkw}{\texttt{lookup}}
\newcommand{\lookup}[2]{\lookupkw #1 \braced{#2} \texttt{else}}
\newcommand{\discard}[1]{#1\texttt{*}}
\newcommand{\substitution}[1]{\texttt{\{}#1{\}}}
\newcommand{\subst}[2]{#1\texttt{=}#2}
\newcommand{\tm}[1]{\texttt{#1}\;}
\newcommand{\pair}[2]{\listof{#1 \pipe #2}}
\newcommand{\wildcard}{\textunderscore}
\newcommand{\lose}[1]{\texttt\##1}
\newcommand{\then}{\texttt{.}\;}
\newcommand{\send}[2]{\channel{#1}\bang#2}
\newcommand{\recv}[2]{\channel{#1}\query#2}
\newcommand{\many}[1]{#1$^*$}
\newcommand{\arr}{\texttt{->}}
\newcommand{\ty}[1]{\tm{#1}}
\newcommand{\compose}[2]{#1 $|$ #2}
\newcommand{\actor}[1]{\texttt{#1}\;}
\newcommand{\spawn}[2]{\judgement{#1}\!\texttt{@}\channel{#2}}
\newcommand{\constrain}[2]{#1 $\sim$ #2}
\newcommand{\syntaxcat}[1]{\textquotesingle\texttt{#1}\;}
\newcommand{\judgement}[1]{\texttt{#1}\;}

% Common variable names
\newcommand{\head}{\tm{head}}
\newcommand{\tail}{\tm{tail}}
\newcommand{\term}{\tm{term}}
\newcommand{\body}{\tm{body}}
\newcommand{\subterm}{\tm{subterm}}
\newcommand{\x}{\tm{x}}
\newcommand{\f}{\tm{f}}
\newcommand{\variable}{\tm{variable}}
\newcommand{\pattern}{\tm{pattern}}
\newcommand{\act}{\actor{actor}}
\newcommand{\actorvariable}{\tm{actor-variable}}

% STLC Syntax
\renewcommand{\Check}{\syntaxcat{Check}}
\newcommand{\Synth}{\syntaxcat{Synth}}
\newcommand{\Type}{\syntaxcat{Type}}
\newcommand{\Lam}{\atom{\texttt{Lam}}}
\newcommand{\Emb}{\atom{\texttt{Emb}}}
\newcommand{\Rad}{\atom{\texttt{Rad}}}
\newcommand{\App}{\atom{\texttt{App}}}
\newcommand{\Nat}{\atom{\texttt{Nat}}}
\newcommand{\Arr}{\atom{\texttt{Arr}}}

% Syntax description syntax
\newcommand{\Syntax}{\texttt{\atom{Syntax}}}
\newcommand{\Tag}{\texttt{\atom{Tag}}}
\newcommand{\Rec}{\texttt{\atom{Rec}}}
\newcommand{\Atom}{\texttt{\atom{Atom}}}
\newcommand{\Nil}{\texttt{\atom{Nil}}}
\newcommand{\Cons}{\texttt{\atom{Cons}}}
\newcommand{\NilOrCons}{\texttt{\atom{NilOrCons}}}
\newcommand{\Bind}{\texttt{\atom{Bind}}}
\newcommand{\Fix}{\texttt{\atom{Fix}}}
\newcommand{\Enum}{\texttt{\atom{Enum}}}
\newcommand{\Term}{\texttt{\atom{Term}}}

% STLC commands
\renewcommand{\check}{\judgement{check}}
\newcommand{\synth}{\judgement{synth}}
\newcommand{\type}{\judgement{type}}

\author{
  Guillaume Allais\inst{1}
  \and
  Malin Altenm\"uller\inst{2}
  \and
  Conor McBride\inst{2}
  \and
  Georgi Nakov\inst{2}
  \and
  Fredrik Nordvall Forsberg\inst{2}
  \and
  Craig Roy\inst{3}
}

\institute{
University of St Andrews,
Fife, United Kingdom
\and
University of Strathclyde,
Glasgow, United Kingdom
\and
Quantinuum \& Cambridge Quantum,
Cambridge, United Kingdom
}

\authorrunning{G. Allais, M. Altenm\"uller, C. McBride, G. Nakov, F. Nordvall Forsberg, and C. Roy}
\titlerunning{\TypOS}

\title{\TypOS: An Operating System for Typechecking Actors}

\begin{document}
\maketitle

% For example:
% - untyped lambda calculus
% - bidi stlc + evalutation

\paragraph{Introduction}
We report our work in progress on \TypOS{}, a domain-specific language
for experimenting with typecheckers and elaborators. %Presented as
%concurrent actors, they share a store of metavariables which become
%monotonically more defined as is typical in type inference.
We tackle typing tasks by spawning a network of
actors~\cite{hewitt1973actors}, each responsible for one node of the
syntax tree, communicating with its parent and child
nodes via channels. Our goals are similar to those of other
domain-specific languages such as Andromeda~\cite{andromeda}, PLT
Redex~\cite{redex}, or Turnstyle+~\cite{turnstyleplus}.

This work is originally motivated by difficulties implementing
elaborators via shallow embedding with opaque continuations, wherein
the values locked in a resumption's closure may
underapproximate what is known by the time it is invoked.
We are effectively forced to choose a first-order representation of
suspended elaboration processes by the need to update them. \TypOS{}
is our attempt to make a virtue of that necessity.

\paragraph{Term syntax and judgement forms}
\TypOS{} supports a generic syntax for terms with LISP-style atoms
\atom{\textit{name}}\!, cons lists $\listof{a_1\, \ldots\, a_n}$, a
distinct $\alpha$-invariant notion of term variables $x$ and binding
$\bind x t$. We provide means to name subsets of the generic syntax
which adhere to formats specified by mutual recursion.  The notion of
\emph{judgement form} is recast as channel \emph{interaction
  protocol}~\cite{honda1993session}, specifying not only what is to be
communicated, but also in which direction.  We \emph{declare} an actor
by giving it a name, and specifying the protocol it follows. For
example, we may readily declare a type checking actor
\[
  \mathtt{check \hab \query\atom{Type}\then \query\atom{Check}\then}
\]
which receives a type and a checkable term and then succeeds or fails,
and a type synthesis actor
\[
  \mathtt{synth \hab \query\atom{Synth}\then \bang\atom{Type}\then}
\]
which receives a synthesizable term, then outputs a type or fails.

\paragraph{Typing rules}
Typing rules are then given by \emph{defining} actors to
perform the communications corresponding to the rules' conclusions. An
actor starts with only the channel to its parent actor, but where
rules have premises, a typing actor spawns children, each with its own
channel.  %Each channel fixes its own required order of communication, but
%communications on different channels may be arbitrarily interleaved.
%Indeed, actors may fork into concurrent subprocesses which communicate
%independently.
Actors may fork, generate fresh metavariables, declare constraints, bind
local variables, match on terms using a
scope-aware pattern language, and push and look up data associated to
local variables.
%
For example, here are actors for bidirectional type checking and type synthesis of the simply typed lambda calculus, following the interaction protocol above:
%
\begin{lstlisting}[columns=fullflexible,keepspaces=true]
check@p = p?ty. p?tm. case tm    -- inputs type and term via p. match on tm.
  { ['Lam \x. body] -> ?S. ?T.   -- lambda case: make fresh metavars S and T.
      ( ty ~ ['Arr S T]          -- unify ty with arrow type. in parallel:
      | \x.                      -- bind fresh x (now in scope for children),
        synth { x -> S }.        -- link S with x for descending synth actors,
        check@q. q!T. q!body. )  -- spawn a child channel q to check the body.
  ; ['Emb e] ->                  -- embedded term e:
      synth@q. q!e. q?S.         -- spawn a child to synthesize type S for e,
      S ~ ty }                   -- unify S and ty.
\end{lstlisting}

\begin{lstlisting}[columns=fullflexible,keepspaces=true]
synth@p = p?tm.                    -- input term tm via p
	  lookup tm { S -> p!S. }  -- try to lookup tm in synth context
	  else case tm
   { ['Ann t ty] ->                -- type annotated term: in parallel
        ( type@q. q!ty.            --   spawn child to validate type ty
        | check@r. r!ty. r!t.      --   spawn child to check t has type ty
        | p!ty. )                  --   send type ty
   ; ['App f s] -> ?U. ?V.     -- application: make fresh metas U and V
        ( synth@q. q!f. q?ty.  -- spawn child to synthesize a type for f
         ty ~ ['Arr U V]       -- unify f with arrow type
        | check@r. r!U. r!s.   -- spawn a child to check argument
        | p!V. ) }             -- send the target type
\end{lstlisting}


\paragraph{Schematic variables and scope management}
The notion of \emph{schematic variable} in a typing rule is recast as
the ordinary notion of program variable within an actor. We are
careful to distinguish these `actor variables' from both `term
variables' of the syntax being manipulated, %by the actors,
and `channel variables' used in input and output actions. Schematic
variables in typing rules are usually thought of as implicitly
universally quantified: by contrast, each of our actor variables has
one explicit binding site in an actor process, either at an input
action, or in a pattern-match. %, and zero or more use sites in expressions.
The term variables in scope for each actor variable are
thus determined precisely~\cite{codebruijn}.
Likewise, the term variables in scope at
the time a channel is created limit what is in scope for the signals
on the channel. When an actor forks in two, its actor and term
variables remain in scope for both parts, but its channel variables
are partitioned between them.

%The input/output distinction aligns with a pattern/expression
%separation seldom made explicit in typing rules, but often vital in
%practice. The expressions we may send as outputs are generous, in
%particular allowing \emph{substitution} of term variables. Patterns,
%meanwhile, analyse inputs in ways which are restricted to ensure that
%actors can access information which is realistically attainable (i.e.,
%not by inverting substitution) and to which they are entitled. By
%contrast, recall the use of substitution in typing rules, e.g. when
%constructing dependent pairs or applying dependent functions, and
%observe that their algorithmic feasibility happens more by luck than
%judgement.

\paragraph{Metavariables and constraints}
The actor model principle of sharing by message rather than memory
allows us to guarantee that each actor has direct knowledge of only
those term variables it has bound itself. Actors have no knowledge of
any term variables which were already free when they were spawned, and
are thus less likely to violate stability under substitution.
%
None the less, we do support the one form of memory that distributed
concurrent processes may safely share~\cite{kuper2015theses} --- \emph{metavariables}, framed
as inputs from a globally shared oracle, mutating only by becoming
defined as the result of a constraint. Metavariables thus provide a
secondary means for actors to communicate. Actors cannot detect
whether a metavariable has been solved, but they can \emph{block} when
they match strictly on a metavariable which is currently unknown, then
unblock subsequently, as the result of a constraint
elsewhere. Decisions taken on the basis of less information need
never be retracted when more arrives.

\paragraph{Executing actors}
We currently implement running actors in sequence until they get stuck
using a stack-based machine; the final configuration of the stack
represents a typing derivation of the term. In the future, we hope to
make better use of the parallelism inherent in the actor model.


\paragraph{Try it yourself}
\TypOS is available at \url{https://github.com/msp-strath/TypOS}, together with more examples of actors.


% this is a WIP
% for prototyping type checkers and elaborators
% based on problems the authors have run into?
% haskell closures - vs first order processing
% high level explanation of what the "OS" means

% \section*{The Actor Model}
% % PT I - Georgi
% % Concurrency
% % Blocking on metas
% % Communication via channels (including metas)
% % Judgement declarations and their interaction protocols

% The actor model is the backbone of concurrent multiprocess communication in \TypOS. % delete/reword
% Actors may spawn other actors and establish parent-child communication via channels.
% Judgement forms then specify what constitutes a valid message exchange, de facto 
% playing the role of a simple non-branching interaction protocol. We provide
% another communication mechanism via a supply of shared metavariables. Actors may 
% define metavariables as the result of a constraint or perform match on a metavariable,
% with the execution of the actor being suspended as long as the metavariable is unsolved. 



% PT II - Guillaume
% The different kinds of actors
% Store of metas w/ news propagation (monotonicity)
% Local scope of actors and terms
% Explain the syntax of whatever example we use

% The language of actors is centered around three key concepts:
% processes and communications,
% terms and pattern-matching,
% metavariables and constraint solving.
% %
% The term-manipulation fragment lets actors bind local variables to go
% under binders and match on terms using a rich scope-aware pattern language
% that can identify binders, or insist that a term does not mention a particular
% local variable.
%
%For instance, the pattern ($\backslash$ x.$\lbrace$x$\star\rbrace$ b)
%corresponds to a binder-headed term ($\backslash$ x.) whose bound variable is not
%used ($\lbrace$x$\star\rbrace$) in the body b.
%

% Substitutions in term syntax

% \section*{Conclusion} % Craig
% % Actual concurrency
% We present a work in progress with much more to be done. Our machine neglects to exploit the parallelism inherent in the actor model, instead running each actor in sequence until they get stuck. Remedying this will enable \TypOS as a platform for writing high performance, highly parallel programming language elaborators.

% % Dependent types
% Our notion of constraints also needs refinement. Currently, the user can write \constrain{\ty{S}\!}{\ty{T}}, to demand that \ty{S} and \ty{T} are resolved via structural unification, but we offer no notion of definitional equality which restricts the type theories which can be expressed.

% % More refined models written in Agda first, then ported
% Our formulation of terms involves different notions of scope requiring many invariants to maintain.
% We mitigate this by prototyping in Agda. Its more expressive type system allows us to offload cognitive overhead by encoding invariants at the type level, which we port to Haskell for the sake of efficient representation. Future iterations may move more work into a more expressive type system such as Agda's for the sake of ease of development and correctness of implementation.

% % Everything is available on github
% \TypOS is available at \url{https://github.com/msp-strath/TypOS}. Along with the source code, we provide a user manual and a handful of examples we have created over the course of development.

\bibliographystyle{plain}
\bibliography{typos}

\end{document}

\TypOS is an operating system for typechecking processes.
Are the processes being typechecked? Are the processes doing the
typechecking? Yes.

\TypOS is a domain-specific language for implementing
type systems, based on an actor model (caveat emptor) of concurrent
execution. Each region of source code is processed by an actor,
implementing a typing judgement with a clearly specified interaction
protocol. Actors may spawn other actors: child processes which
correspond to the premises of typing rules. Actors communicate with
one another along channels, in accordance with their stated protocol.

A \TypOS program consists of
\begin{itemize}
\item descriptions of syntaxes
\item declarations of judgement interaction protocols
\item definitions of judgement actors
\item example invocations of judgement actors
\end{itemize}

And when you press go, you get to watch the execution trace of your
examples, as control flits about between a treelike hierarchy of
actors, making progress wherever possible. We are still at a
rudimentary stage: there is a lot which remains to be done.


\subsection*{What gets typechecked?}

We have a very simple syntax for the stuff that gets processed by
\TypOS actors.

A \emph{name} begins with an alphabetical character, followed by zero or
more characters which are alphabetical, numeric or \wildcard\,. Names are
used in two ways:

\begin{itemize}
\item as variables, \x, once they have been brought into scope
\item as atoms, \atom{\x}, whose role is to be different from each other
\end{itemize}

We build up structure by means of

\begin{itemize}
\item binding: \bind{\x} \term\; is a term which brings \x\; into scope in \term
\item pairing:
\pair{\term}{\term}
is a term with two subterms
\end{itemize}

There is a special atom, \nil\; (pronounced ``nil''), and we adopt the
bias towards right-nesting which has been prevalent since LISP
established it in the 1950s. That is, every occurrence of
\texttt{|[} may be removed, provided its corresponding \rsq is removed also.
It is typical, therefore, to build languages with terms of the form
\listof{\atom{\tm{tag}} \subterm .. \subterm}.

Now, there is no particular ideological significance to this choice of
LISP-with-binding. It is a cheap way to get started, but it is not an
ideal way to encode languages with more complex scoping
structure. When we need more, we shall review this choice.

Of course, LISP-with-binding is intended merely as a substrate: not
all terms are expected to be meaningful in all situations. We provide
various means of classification. Let us begin.


\subsubsection*{\texttt{syntax} declarations}

We start with a basic notion of context-free grammar. A \syntax
declaration allows a bunch of nonterminal symbols to be mutually
defined. Here is an example, being a bidirectional presentation of
simply typed lambda-calculus.

\begin{verbatim}
syntax
  { 'Check = ['Tag  [ ['Lam ['Bind 'Synth ['Rec 'Check]]]
                      ['Emb ['Rec 'Synth]]
             ]]
  ; 'Synth = ['Tag  [ ['Rad ['Rec 'Check] ['Rec 'Type]]
                      ['App ['Rec 'Synth] ['Rec 'Check]]
             ]]
  ; 'Type = ['Tag  [ ['Nat ]
                     ['Arr  ['Rec 'Type] ['Rec 'Type]]
            ]]
  }
\end{verbatim}

What are we saying? The terms you can \Check can be
lambda-abstractions but we also \Emb ed all the terms whose type we
can \Synth esize. The latter comprise what we call \Rad icals
(checkable terms with a type annotation) and \App lications where a
synthesizable function is given a checkable argument. E.g., the
identity function might be written
\listof{\Lam \bind{x} \listof{\Emb x}}
.

How have we said it? Following the keyword \syntax is a
\braced{;}-list
of equations, each defining an atom by a term which happens to be a
\emph{syntax description}. You can see some components of syntax
descriptions:

\begin{itemize}
\item \listof{\Tag ..} takes a list of pairs of atoms with lists of syntax
  descriptions, allowing us to demand exactly a list whose head is a
  tag and whose other elements are specified in a manner selected by
  the tag. So \Lam and \Emb are the tags for \Check terms,
  \Rad and \App for \Synth terms, \Nat and \Arr for \Type
  terms.
\item \listof{\Rec ..}, for \emph{recursive}, is followed by an atom which is the name of
  a syntax, including the syntaxes defined in previous syntax
  declarations, or the current syntax declaration. E.g., we see that
  the \Emb tag should be followed by one \Synth term, while the
  \Arr tag should be followed by two \Type's.
\item \listof{\Bind ..} specifies a term of form \bind{\tm x}
  It takes an
  atom which determines the named syntax to which the \x is being
  added, then a syntax description for the \texttt{t}.
\end{itemize}

Correspondingly, in our example above, the \x is classified as a
\Synth term, and so must be \Emb edded as the \Check able body of
the \Lam bda abstraction.

The other permitted syntax descriptions are as follows:

\begin{itemize}
\item \listof{\Nil} admits only \nil.
\item \listof{\Cons \head \tail} admits pairs \pair{\tm{h}}{\tm{t}} where \head admits \tm{h} and \tail admits \tm{t}.
\item \listof{\NilOrCons \head \many{\tail}} admits the union of the above two.
\item \listof{\Atom} admits any atom.
\item \listof{\Enum ..} takes a list of atoms and admits any of them. Note that \listof{\Enum \pair{\atom{\tm{tt}}{\tm{ff}}}} admits \tm{tt} and \tm{ff}, as opposed to
\item \listof{\Tag \listof{\listof{\atom{\tm{tt}}}\listof{\atom{\tm{ff}}}}}, which admits \listof{\atom{\tm{tt}}} and \listof{\atom{\tm{ff}}}.
\item \listof{\Fix \bind{\x} ..} takes a syntax decsription in which the bound
\item \x is treated as a syntax description, allowing local recursion.
\item \listof\Term admits anything.
\end{itemize}

For a more exciting example, we take
\begin{verbatim}
syntax { 'Syntax = ['Tag [
  ['Rec ['Atom]]
  ['Atom]
  ['Nil]
  ['Cons ['Rec 'Syntax] ['Rec 'Syntax]]
  ['NilOrCons ['Rec 'Syntax] ['Rec 'Syntax]]
  ['Bind ['Atom] ['Rec 'Syntax]]
  ['Tag
    ['Fix (\list.['NilOrCons
                   ['Cons ['Atom]
                   ['Fix (\list.['NilOrCons ['Rec 'Syntax] list])]
                 ] list])]]
  ['Fix ['Bind 'Syntax ['Rec 'Syntax]]]
  ['Enum ['Fix (\list.['NilOrCons ['Atom] list])]]
  ['Term]
]]}
\end{verbatim}

as the syntax description of syntax descriptions, using \Fix to
characterize the lists which occur in the \listof{\Tag ..} and
\listof{\Enum ..} constructs.


\subsubsection*{Judgement forms and protocols}

Before we can implement the actors which process our terms, we must
say which actors exist and how to communicate with them. Our version
of Milner's judgement-form-in-a-box names is to declare
\emph{name} $:$ \emph{protocol}. A protocol is a sequence of \emph{actions}. Each
action is specified by \query for input or \bang for output, then the
intended syntax for that transmission, then \then as a closing
delimiter.

For our example language, we have

\begin{verbatim}
type  : ?'Type.
check : ?'Type. ?'Check.
synth : ?'Synth. !'Type.
\end{verbatim}

indicating that \type actors receive only a \Type which they may
validate; \check actors receive a \Type to check and a \Check able
term which we hope the type admits; \synth actors receive a
\Synth esizable term, then (we hope) transmit the \Type synthesized
for that term.

Our protocols are nowhere near as exciting as session types, offering
only a rigid sequence of actions to do (or die). In the future, we
plan to enrich the notion of protocol in two ways:

\begin{enumerate}
\item Designate one input as the \emph{subject} of the judgement, i.e., the
   currently untrusted thing whose validity the judgement is intended
   to establish. Above, the clue is in the name.
\item For every signal which is not the subject, indicate the
   \emph{contract}. The actor is allowed to \emph{rely} on properties of its
   inputs, but it must \emph{guarantee} properties of its outputs. For the
   above, we should let \check rely on receiving a \Type which
   \type accepts, but demand that \synth always yields a \Type
   which \type accepts.
\end{enumerate}

That is, we plan to check the checkers: at the moment we just about
check that actors stick to the designated interleaving of input and
output operations. The syntax for each signal is but a good intention
upon which we do not yet act.


\subsubsection*{\TypOS actors}

An actor definition looks like \judgement{judgement}@\channel{channel} = \act.
The channel is the actor's link with its parent (so we often call it
\channel{p}) along which it must follow the declared protocol.
Here is a simple example:

\begin{verbatim}
type@p = p?ty. case ty
  { ['Nat] ->
  ; ['Arr S T] ->
      ( type@q. q!S.
      | type@r. r!T.
      )
  }
\end{verbatim}

This actor implements \type, with channel \channel{p} to its parent. Its
first action is \send{p}{\ty{ty}}\then to ask its parent for an input, which
comes into scope as the value of the \actorvariable \ty{ty}\then I.e.,
a \emph{receiving} actor looks like
\recv{channel}{\actorvariable}\then \act, which
performs an input on the given \emph{channel}, then continues as the
\act with the \actorvariable in scope. Actor variables stand for
terms, and may be used in terms as placeholders. Our actor has
received a type to validate. How does it proceed?

It performs a \case analysis on the structure of the type. The
actor construct is
{\case \term $\lbrace$ \pattern \arr \act \texttt{; ..} $\rbrace$}
. We shall specify patterns in more detail shortly, but let us
continue the overview. The \listof{\Nat} pattern matches only if \ty{ty}
is exactly \listof\Nat, and the action taken in that case is nothing
at all! The empty actor denotes glorious success! Meanwhile, the
pattern \listof{\Arr \ty{S} \ty{T}} matches any three element list whose head is
the atom \Arr: the other two elements are brought into scope as
\ty{S} and \ty{T}, repsectively, then we proceed with the nonempty actor
to the right of \arr. What have we, now?

We have a \emph{parallel} composition, \compose{\act}{\act}, and both
components will run concurrently. The first begins by \emph{spawning}
a new \type actor on fresh channel \channel{q}. Spawning looks like
\spawn{judgement}{channel}\then \act, and it is another sequential
process, forking out a new actor for the given judgement and naming
the \emph{channel} for talking to it, before continuing as the given
\emph{actor} with the \emph{channel} in scope. The channel follows the
protocol \emph{dual} to that declared for the judgement. Our first fork
continues by \emph{sending} \ty{S} to \channel{q}. Sending looks like
\send{channel}{\term}\then \act. That is, we have delegated the
validation of \ty{S} to a subprocess and hung up our boots, contented.
The second fork similarly delegates the validation of \ty{T} to another
\type actor on channel \channel{r}.

We have seen actors for receiving, sending, case analysis, parallel
composition, and spawning. There is a little more to come. Let us
have a further example:

\begin{verbatim}
check@p = p?ty. p?tm. case tm
  { ['Lam \\x. body] -> ?S. ?T.
      ( ty ~ ['Arr S T]
      | \\x. synth { x -> S }. check@q. q!T. q!body.
      )
  ; ['Emb e] -> synth@q. q!e. q?S. S ~ ty
  }
\end{verbatim}

The \check actor follows the designated protocol, asking its parent
for a type \ty{ty} and a checkable term \tm{tm}. We expect \tm{tm} to match
one of two patterns. The second is the simpler \listof{\Emb \tm{e}} matches an
embedded \Synth term, bound to \tm{e}, then spawns a \synth actor
on channel \channel{q} to determine the type of \tm{e}. That is, we send \tm{e} over
\channel{q}, then receive type \ty{S} in return. Our last act in this case is to \emph{constrain}
\constrain{\ty{S}}{\ty{ty}}
, i.e., we demand that the type synthesized is
none other than the type we were asked to check. The actor form \constrain{\term}{\term}
performs a unification process, attempting to make the
terms equal.


The \listof{\Lam \bind{\x} \body} case shows a richness of features. Firstly,
the pattern indicates that the term must bind a variable, which the
term can name however it likes, but which the actor will think of as
\x . The pattern variable \body matches what is in the scope of \x .
As a consequence, \body stands for a term which may mention \x and
thus may be used only in places where \x is somehow captured. That
is, the use sites of actor variables are scope-checked, to ensure that
everything the terms they stand for might need is somehow in
existence. We have found the body of our abstraction. What happens
next?

It looks like we are making inputs \ty{S} and \ty{T} from \emph{no} channel, and
that is exactly what we are doing! We request \ty{S} and \ty{T} from \emph{thin
air}. Operationally, TypOS generates placeholders for terms as yet
unknown, but which may yet be solved, given subsequent constraints.
Indeed, one of our subsequent forked actors exactly demands that \ty{ty}
is \listof{\Arr \ty{S} \ty{T}}, but we need not wait to proceed. In parallel, we
\emph{bind} a fresh variable \x, allowing us to spawn a \check actor on
channel \channel{q} and ask it to check that type \ty{T} admits \body (whose \x
has been captured by our binding). But we race ahead. A \emph{binding}
actor looks like \bind{\variable}\act. It brings a fresh term
\variable into scope, then behaves like \act for the duration of
that scope.

Now, before we can \check the \body, we must ensure that \synth
knows what to do whenever it is asked about \x. We have explored
various options about how to manage that interaction. The current
incarnation is to equip each judgement with its own notion of
\emph{contextual data} for free variables. The form
\tm{judgement}\braced{\variable \arr \term}\then \act
pushes the association of \term with \variable into the context for
\tm{judgement}, then continues as \act. In our example, we have
\synth \braced{ \x \arr \ty{S} }\then \spawn{check}{q}\then \send{q}{\ty{T}}\then \send{q}{\body}\then
, so any \synth actor
which is a descendant of the \check actor on channel \channel{q} will
be able to access the \ty{S} associated with \x . To see how, we must
look at the \synth actor's definition.

\begin{verbatim}
synth@p = p?tm . lookup tm { S -> p!S. } else case tm
  { ['Rad t ty] ->
      ( type@q. q!ty.
      | check@r. r!ty. r!t.
      | p!ty.
      )
  ; ['App f s] -> ?U. ?V.
      ( synth@q. q!f. q?ty. ty ~ ['Arr U V]
      | check@r. r!U. r!s.
      | p!V.
      )
  }
\end{verbatim}

We have only one new feature, which is invoked immediately we have
received \tm{tm}. The actor \lookup{\term}{\actorvariable \arr \act} \act attempts to access the context for the judgement it
is implementing, i.e. \synth. It will succeed if \tm{tm} stands for a
free term variable with a context entry in scope, and in that case,
the \actorvariable binds the associated value and the \act after \arr
is executed. As you can see, \synth interprets the contextual data
associated with a free variable as exactly the type to send out.
If the \term is not a free variable, or if there is no associated
data in the context, the \lookupkw actor falls through to its \texttt{else}
clause.

Here, we fall back on the hope that \tm{tm} might take one of the
two forms specified in the syntax of \Synth terms. For \Rad icals, we
concurrently validate the type, check that it accepts the term, and
deliver the type as our output. For \App lications, we guess source
and target types for the function, then concurrently confirm our guess
by constraining the output of \synth on the function, check the
argument at our guessed source type, and output our guessed target
type as the type of the application.

You have been watching
\begin{itemize}
\item guessing: \query\actorvariable\then \act
\item receiving: \recv{\channel{channel}}{\actorvariable}\then\act
\item sending: \send{\channel{channel}}{\actorvariable}\then\act
\item casing: \case \term % `{` *pattern* `->` *actor* `;` ..`}`
\item forking: \compose{\act}{\act}
\item spawning: \spawn{judgement}{channel}\then \act
\item constraining: \constrain{\term}{\term}
\item pushing: \texttt{judgement} \braced{\variable \arr \term}\then \act
\item looking: \lookup{\term}{\actorvariable \arr \act} \act
\item winning:
\end{itemize}

and there's one more
\begin{itemize}
\item losing: \lose{\tm{string}}
\end{itemize}

\subsubsection*{Patterns}

The patterns you can write in a \TypOS \case actor look like
\begin{itemize}
\item term variable:    \variable
\item specific atom:    \atom{\tm{name}}
\item paired patterns:  \pair{\pattern}{\pattern}
\item variable binding: \bind{\variable}\pattern
\item scope selection:  \braced{\texttt{selection}} \pattern
\item pattern binding:  \actorvariable
\item happy oblivion:   \wildcard
\end{itemize}

where a \emph{selection} (sometimes known dually as a \emph{thinning}) selects a
subset of the variables in scope as permitted dependencies. Inside the
braces, you write either the list of variables you want to keep or the
list of variables you want to exclude then \discard{}.

A term variable pattern matches exactly that term variable: we can
tell them apart from the pattern bindings of actor variables because
we can see where the term variable has been bound, whereas the actor
variable for a pattern binding must be fresh. E.g., we might want to
spot eta-redexes in our lambda calculus with the pattern
\listof{\Lam\bind{\x} \listof{\Emb \listof{\App \substitution{\discard\x}\f \x}}}
. The \f is a pattern binding and
it matches any term not depending on \x. The \x is a bound variable,
and it matches only the variable bound by the \bind{\x}.

At the binding sites of actor variables, TypOS detects which term
variables are in scope. TypOS further insists that everything which
was in scope at an actor variable's binding site is in scope at each
of its use sites. In the above example, we check that \f matches a
term in which \x plays no part, and we gain the right to use \f with
no \x in scope.


\subsection*{Substitutions}

When we write terms in actors, we are really talking \emph{about} terms,
with actor-variables standing for terms, generically. Now, we have
insisted that every term variable in scope at an actor-variable's
binding site must be captured at each of its use sites, and we have
seen that one way to do that is with another \bind\variable binding
with a matching name. However, we may also \emph{substitute} such
variables.

We extend the syntax of terms with \substitution{\tm{substitution}}\term. A
substitution is a comma-separated list of components which look like
\begin{itemize}
\item definition: \texttt{\variable = \term}
\item exclusion: \discard{\variable}
\item preservation: \texttt{\variable}
\end{itemize}

Order matters: substitutions should be read from right to left as
actions on the scope we find them in. Definitions bring new variables
into scope, by defining them to be terms using only variables already
in scope. Exclusions throw variables out of scope. Preservations
retain variables from the present scope but only as their own images:
in \substitution{\subst{y}{t},\x}, the \tm{t} may not depend on \x. A leftmost prefix of
preservations may safely be elided, so we need only say what is
\emph{changing} at the local end of scope.

We may write substitutions anywhere in a term, but they act
structurally on all the term constructs (acquiring an extra
preservation at the local end wherever they go under a binder), piling
up at the use sites of actor-variables, where their role is to
reconcile any difference in scope with the binding sites of those
variables.


\subsection*{Actors, channels, scope}

Each actor knows about only those variables it binds itself. When
actors run, the terms which the actor-variables stand for will be
in a larger scope: the term variables mentioned in the actor's source
code will constitute the local end of that scope. Although the
\lookupkw construct enables the basic means to find out stuff about
free variables, only the actor which binds the variable can choose
what that stuff is. Ignorance of free variables makes it easier to
achieve stability under substitution. In particular, the fact that
\case patterns can test for only those free variables protected by their
binders from the action of substitution means that an actor's
\case choices cannot be changed by the action of substitution on its
inputs. There is some sort of stability property to be proven about
\texttt{\lookupkw .. else}, characterizing the things it is safe to substitute
for free variables.

Meanwhile, channels also have a notion of scope, restricting the
variables which may occur free in the terms which get sent along
them. The scope of a channel is exactly the scope at its time of
creation.
\end{document}
